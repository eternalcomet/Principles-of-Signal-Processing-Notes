\subsection{习题课 3}

\begin{exercise}
    计算:
    \begin{enumerate}[label=(\arabic*)]
        \item 已知序列 $x_1 = [1, 6, 5, 3], x_2 = [2, 7, 5, 4, 0, 1]$,
            求它们的线卷积和 $6$ 点圆卷积。
        \item 设序列 $x(n) = [1, 2, 6, 3]$,$x(n)$ 的 $4$ 点离散傅里叶
            变换 DFT 为 $X(k)$,求 $X^2(k)$ 的 $4$ 点离散傅里叶逆变换 IDFT。
    \end{enumerate}
\end{exercise}

\begin{solution}
    \begin{enumerate}[label=(\arabic*)]
        \item 线卷积公式为 $y(n) = \sum_{m = -\infty}^{+\infty}x_1(m)x_2(n - m)$,即
            \begin{figure}[H]
                \centering
                \begin{tabular}{c c c c c c c c c c}
                    & & & & 1 & 0 & 4 & 5 & 7 & 2 \\
                    $\times$ & & & & & & 3 & 5 & 6 & 1 \\
                    \hline
                    & & & & 1 & 0 & 4 & 5 & 7 & 2 \\
                    & & & 6 & 0 & 24 & 30 & 42 & 12 \\
                    & & 5 & 0 & 20 & 25 & 35 & 10 \\
                    $+$ & 3 & 0 & 12 & 15 & 21 \\
                    \hline
                    & 3 & 5 & 18 & 36 & 70 & 75 & 57 & 19 & 2
                \end{tabular}
            \end{figure}
            因此,线卷积为 $y(n) = [2, 19, 57, 75, 70, 36, 18, 5, 3]$。
            
            而 $6$ 点圆卷积为线卷积的回绕,即
            \begin{align*}
                z(n) = x_1(n) \otimes x_2(n) = \rev{x_1(n) * x_2(n)}.
            \end{align*}
            \begin{figure}[H]
                \centering
                \begin{tabular}{c c c c c c c}
                    & 36 & 70 & 75 & 57 & 19 & 2 \\
                    $+$ & & & & 3 & 5 & 18 \\
                    \hline
                    & 36 & 70 & 75 & 60 & 24 & 20
                \end{tabular}
            \end{figure}
            因此,$6$ 点圆卷积为 $z(n) = [20, 24, 60, 75, 70, 36]$。
        \item 由 IDFT 的卷积定理,有 $\IDFT{X^2(k)} = x(n) \otimes x(n)$,
            故只需要计算 $x(n)$ 与自己圆卷积即可。先计算 $x(n)$ 与自己的线卷积:
            \begin{figure}[H]
                \centering
                \begin{tabular}{c c c c c c c c}
                    & & & & 3 & 6 & 2 & 1 \\
                    $\times$ & & & & 3 & 6 & 2 & 1 \\
                    \hline
                    & & & & 3 & 6 & 2 & 1 \\
                    & & & 6 & 12 & 4 & 2 \\
                    & & 18 & 36 & 12 & 6 \\
                    $+$ & 9 & 18 & 6 & 3 \\
                    \hline
                    & 9 & 36 & 48 & 30 & 16 & 4 & 1
                \end{tabular}
            \end{figure}
            然后计算 $x(n)$ 与自己的圆卷积:
            \begin{figure}[H]
                \centering
                \begin{tabular}{c c c c c}
                    & 30 & 16 & 4 & 1 \\
                    $+$ & & 9 & 36 & 48 \\
                    \hline
                    & 30 & 25 & 40 & 49
                \end{tabular}
            \end{figure}
            因此,$X^2(k)$ 的 $4$ 点离散傅里叶逆变换 IDFT 为 $[49, 40, 25, 30]$。
    \end{enumerate}
\end{solution}

\begin{note}
    在进行竖式计算的时候,一定\bd{不要进位}!这是在计算卷积,
    而不是在进行竖式乘法!

    还有一点要注意的是,进行竖式计算时,序列是按照从低位到高位的顺序填入的,
    最终的结果也是按照从低位到高位的顺序排列的。所以最后要将结果\bd{逆序排列}。
\end{note}

\begin{exercise}
    设序列 $x_1(n), x_2(n)$ 的长度分别为 $L_1, L_2$,$N$ 点离散傅里叶
    变换 DFT 分别为 $X_1(k), X_2(k)$。
    \begin{enumerate}[label=(\arabic*)]
        \item 记 $x_1(n)$ 和 $x_2(n)$ 的 $N$ 点圆卷积为 $y(n)$,求证:
            \begin{align*}
                \sum_{n = 0}^{N - 1}y(n) = \left(\sum_{n = 0}^{L_1 - 1}x_1(n)\right)\left(\sum_{n = 0}^{L_2 - 1}x_2(n)\right).
            \end{align*}
        \item 当 $L_1 = 6, N = 5$ 时,若 $x_1(n) = [2, 5, 1, 0, 3, -2]$,
            求 $X_1^2(k)$ 的 $5$ 点离散傅里叶逆变换 IDFT。
        \item 当 $L_2 = 4, N = 5$ 时,若 $x_2(0) = 2, x_2(2) = \mathi$,
            且 $X_2(k)$ 是实序列,求 $X_2(k)$ 的 $5$ 点离散傅里叶变换 DFT。
    \end{enumerate}
\end{exercise}

\begin{solution}
    \begin{enumerate}[label=(\arabic*)]
        \item 先对 $x_1(n)$ 和 $x_2(n)$ 进行补零或回绕得到长度为 $N$ 的
            序列 $\rev{x_1}(n)$ 和 $\rev{x_2}(n)$。由 DFT 的性质知,
            \begin{align*}
                X_1(k) = \DFT{x_1(n)} = \DFT{\rev{x_1}(n)}, \\
                X_2(k) = \DFT{x_2(n)} = \DFT{\rev{x_2}(n)}.
            \end{align*}
            设 $y(n)$ 的 $N$ 点 DFT 为 $Y(k)$,则
            \begin{align*}
                Y(k) & = \DFT{y(n)} \\
                & = \DFT{x_1(n) \otimes x_2(n)} \\
                & = X_1(k) \cdot X_2(k) \\
                & = \DFT{\rev{x_1}(n)} \cdot \DFT{\rev{x_2}(n)} \\
                & = \DFT{x_1(n)} \cdot \DFT{x_2(n)}
            \end{align*}
            因此,按照 DFT 的定义进行展开,有
            \begin{align*}
                \sum_{n = 0}^{N - 1}y(n)W_N^{kn}
                    = \left(\sum_{n = 0}^{L_1 - 1}x_1(n)W_{L_1}^{kn}\right)
                    \left(\sum_{n = 0}^{L_2 - 1}x_2(n)W_{L_2}^{kn}\right).
            \end{align*}
            取 $k = 0$,即得
            \begin{align*}
                \sum_{n = 0}^{N - 1}y(n) = \left(\sum_{n = 0}^{L_1 - 1}x_1(n)\right)\left(\sum_{n = 0}^{L_2 - 1}x_2(n)\right).
            \end{align*}
            命题得证。
        \item 由 IDFT 的卷积定理,有 $\IDFT{X_1^2(k)} = x_1(n) \otimes x_1(n)$,
            故只需要计算 $x(n)$ 与自己圆卷积即可。先计算 $x(n)$ 与自己的线卷积:
            \begin{figure}[H]
                \centering
                \begin{tabular}{c c c c c c c c c c c c}
                    & & & & & & -2 & 3 & 0 & 1 & 5 & 2 \\
                    $\times$ & & &  & & & -2 & 3 & 0 & 1 & 5 & 2 \\
                    \hline
                    & & & & & & -4 & 6 & 0 & 2 & 10 & 4 \\
                    & & & & & -10 & 15 & 0 & 5 & 25 & 10 \\
                    & & & & -2 & 3 & 0 & 1 & 5 & 2 \\
                    & & -6 & 9 & 0 & 3 & 15 & 6 \\
                    $+$ & 4 & -6 & 0 & -2 & -10 & -4 \\
                    \hline
                    & 4 & -12 & 9 & -4 & -14 & 22 & 13 & 10 & 29 & 20 & 4
                \end{tabular}
            \end{figure}
            然后计算 $x(n)$ 与自己的圆卷积:
            \begin{figure}[H]
                \centering
                \begin{tabular}{c c c c c c}
                    & 13 & 10 & 29 & 20 & 4 \\
                    & -12 & 9 & -4 & -14 & 22 \\
                    $+$ & & & & & 4 \\
                    \hline
                    & 1 & 19 & 25 & 6 & 30
                \end{tabular}
            \end{figure}
            因此,$X_1^2(k)$ 的 $5$ 点离散傅里叶逆变换 IDFT 为 $[30, 6, 25, 19, 1]$。
        \item 设 $x_2(n) = [2, a, \mathi, b]$,其中 $a, b \in \set{C}$。
            由于 $L_2 < N$,故在 $x_2(n)$ 末尾补零,即 $x_2(n) = [2, a, \mathi, b, 0]$。
            由 DFT 的对称性,
            \begin{align*}
                \DFT{X_2(k)} = N\rev{x_2}(-n) = [10, 0, 5b, 5\mathi, 5a].
            \end{align*}
            由于 $X_2(k)$ 为实序列,故其 DFT 共轭对称,即
            \begin{align*}
                \DFT{X_2(k)} = [10, 0, 5b, 5\mathi, 5a] = [10, 5a^*, -5\mathi, 5b^*, 0].
            \end{align*}
            因此 $a = 0, b = -\mathi$,$X_2(k) = [10, 0, -5\mathi, 5\mathi, 0]$。
    \end{enumerate}
\end{solution}

\begin{note}
    在处理 DFT 相关问题的时候,如果序列长度不相等,一定要进行补零或回绕。
    这一步是平凡的,因为进行补零或回绕不会影响 DFT 的结果。

    其次要注意的点是,$\rev{x}(-n)$ 相当于是 $x(0)$ 不变,其余的元素逆序排列;
    共轭对称的时候,$X(0)$ 也是不变的,其余的元素逆序排列对应。
\end{note}

\begin{exercise}
    本课程介绍了``基 $2$ 的 IFFT 算法'',即将 $N$ 点 IDFT 运算递归地
    分解为 $N/2$ 点,$N/4$ 点,……,$2$ 点 IDFT 运算。事实上,也可以将 $N$ 点 IDFT 运算
    递归地分解为 $N/3$ 点,$N/9$ 点,……,$3$ 点 IDFT 运算,实现 ``基 $3$ 的 IFFT 算法''。

    设序列 $X(k)$ 的长度为 $N$,$X(k)$ 的 $N$ 点 IDFT 为 $x(n)$,$N$ 是 $3$ 的倍数。
    将序列 $X(k)$ 分解为 $3$ 个长度为 $N/3$ 的子序列:
    \begin{align*}
        X_0(k) = X(3k), \quad X_1(k) = X(3k + 1), \quad X_2(k) = X(3k + 2), \quad 0 \le k < N/3.
    \end{align*}
    设序列 $X_0(k), X_1(k), X_2(k)$ 的 $N/3$ 点 IDFT 分别
    为 $x_0(n), x_1(n), x_2(n)$,试用 $x_0(n), x_1(n), x_2(n)$ 表示 $x(n)$。
\end{exercise}

\begin{solution}
    将
    \begin{align*}
        x(n) = \frac{1}{N}\sum_{k = 0}^{N - 1}X(k)W_N^{-kn}
    \end{align*}
    分为三部分,即
    \begin{align*}
        x(n) & = \frac{1}{N}\sum_{k = 0}^{N/3 - 1}X(3k)W_N^{-3kn}
            + \frac{1}{N}\sum_{k = 0}^{N/3 - 1}X(3k + 1)W_N^{-(3k + 1)n}
            + \frac{1}{N}\sum_{k = 0}^{N/3 - 1}X(3k + 2)W_N^{-(3k + 2)n} \\
            & = \frac{1}{N}\sum_{k = 0}^{N/3 - 1}X_0(k)W_{N/3}^{-kn}
            + \frac{W_N^{-n}}{N}\sum_{k = 0}^{N/3 - 1}X_1(k)W_{N/3}^{-kn}
            + \frac{W_N^{-2n}}{N}\sum_{k = 0}^{N/3 - 1}X_2(k)W_{N/3}^{-kn} \\
            & = \frac{1}{N}\cdot\frac{N}{3}x_0(n)
            + \frac{W_N^{-n}}{N}\cdot\frac{N}{3}x_1(n)
            + \frac{W_N^{-2n}}{N}\cdot\frac{N}{3}x_2(n) \\
            & = \frac{1}{3}\left(x_0(n) + W_N^{-n}x_1(n) + W_N^{-2n}x_2(n)\right),
    \end{align*}
    其中 $n = 0, 1, \cdots, N / 3 - 1$。
    类似于 FFT 的推导过程,可以得到
    \begin{align*}
        x(n) & = \frac{1}{3}\left(x_0(n) + W_N^{-n}x_1(n) + W_N^{-2n}x_2(n)\right), \\
        x\left(n + \frac{N}{3}\right) & = \frac{1}{3}\left(x_0(n) + W_N^{-(n + N/3)}x_1(n) + W_N^{-2(n + N/3)}x_2(n)\right), \\
        x\left(n + \frac{2N}{3}\right) & = \frac{1}{3}\left(x_0(n) + W_N^{-(n + 2N/3)}x_1(n) + W_N^{-2(n + 2N/3)}x_2(n)\right),
    \end{align*}
    其中 $n = 0, 1, \cdots, N / 3 - 1$。
\end{solution}

\begin{exercise}
    本课程介绍了 DFT 的快速算法 FFT。相应地,可将 $N$ 点 IDFT 运算递归地分解
    为 $N/2$ 点,$N/4$ 点,……,$2$ 点 IDFT 运算,从而实现 IDFT 的快速计算。
    \begin{enumerate}[label=(\arabic*)]
        \item 设序列 $X(k)$ 的长度为 $N$,$N$ 为偶数,$X(k)$ 的 $N$ 点 IDFT 为 $x(n)$。$G(k)$ 是 $X(k)$ 中
            下标为偶数的元素组成的子序列,$H(k)$ 是 $X(k)$ 中下标为奇数的元素组成的子序列。
            它们的长度均为 $N/2$,对应的 $N/2$ 点 IDFT 分别为 $g(n)$ 和 $h(n)$。
            试用 $g(n)$ 和 $h(n)$ 表示 $x(n)$。        
        \item 规定 $X(k)$ 的 $m$ 点子序列 $[X(k_1), X(k_2), \cdots, X(k_m)]$ 的 $m$ 点 IDFT 可
            表示为 $x_{k_1, k_2, \cdots, k_m}(n)$。当 $n = 8$ 时,试利用 (1) 中的结果,
            用若干组 $X(k)$ 的 $2$ 点子序列的 $2$ 点 IDFT 表示 $x(3)$。
    \end{enumerate}
\end{exercise}

\begin{solution}
    \begin{enumerate}[label=(\arabic*)]
        \item IDFT 的定义为
            \begin{align*}
                x(n) = \frac{1}{N}\sum_{k = 0}^{N - 1}X(k)W_N^{-kn},
            \end{align*}
            可以将其分为两组,
            \begin{align*}
                x(n) & = \frac{1}{N}\sum_{k = 0}^{N/2 - 1}X(2k)W_N^{-2kn}
                    + \frac{1}{N}\sum_{k = 0}^{N/2 - 1}X(2k + 1)W_N^{-(2k + 1)n} \\
                & = \frac{1}{N}\sum_{k = 0}^{N/2 - 1}G(k)W_{N/2}^{-kn}
                    + \frac{W_N^{-n}}{N}\sum_{k = 0}^{N/2 - 1}H(k)W_{N/2}^{-kn}.
            \end{align*}
            由于 $G(k) = X(2k), H(k) = X(2k + 1)$,其中 $k = 0, 1, \cdots, N / 2 - 1$,故
            \begin{align*}
                \begin{cases}
                    \sum_{k = 0}^{N/2 - 1}X(2k)W_{N/2}^{-kn}
                        = \sum_{k = 0}^{N/2 - 1}G(r)W_{N/2}^{-kn} = \frac{N}{2}g(n), \\
                    \sum_{k = 0}^{N/2 - 1}X(2k + 1)W_{N/2}^{-kn}
                        = \sum_{k = 0}^{N/2 - 1}H(r)W_{N/2}^{-kn} = \frac{N}{2}h(n),
                \end{cases}
            \end{align*}
            其中 $n = 0, 1, \cdots, N/2 - 1$。因此
            \begin{align*}
                x(n) = \frac{1}{2}\left(g(n) + W_N^{-n}h(n)\right), \quad n = 0, 1, \cdots, N / 2 - 1.
            \end{align*}
            再考虑后半部分,
            \begin{align*}
                x\left(\frac{N}{2} + n\right) & = \frac{1}{N}\sum_{k = 0}^{N/2 - 1}X(2k)W_{N/2}^{-k\left(\frac{N}{2} + n\right)}
                    + \frac{W_N^{-\left(\frac{N}{2} + n\right)}}{N}\sum_{k = 0}^{N/2 - 1}X(2k + 1)W_{N/2}^{-k\left(\frac{N}{2} + n\right)} \\
                & = \frac{1}{N}\sum_{k = 0}^{N/2 - 1}X(2k)W_{N/2}^{-kn}
                    - \frac{W_N^{-n}}{N}\sum_{k = 0}^{N/2 - 1}X(2k + 1)W_{N/2}^{-kn} \\
                & = \frac{1}{N}\sum_{k = 0}^{N/2 - 1}G(k)W_{N/2}^{-kn}
                    - \frac{W_N^{-n}}{N}\sum_{k = 0}^{N/2 - 1}H(k)W_{N/2}^{-kn} \\
                & = \frac{1}{2}\left(g(n) - W_N^{-n}h(n)\right), \quad n = 0, 1, \cdots, N / 2 - 1.
            \end{align*}
            综上所述,
            \begin{align*}
                x(n) = \frac{1}{2}\left(g(n) + W_N^{-n}h(n)\right),
                    \quad x\left(\frac{N}{2} + n\right) = \frac{1}{2}\left(g(n) - W_N^{-n}h(n)\right),
            \end{align*}
            其中 $n = 0, 1, \cdots, N / 2 - 1$。
        \item 由 (1) 知
            \begin{align*}
                x_{0, 1, \cdots, 7}(3) & = x(3) \\
                & = \frac{1}{2}\left(g(3) + W_8^{-3}h(3)\right) \\
                & = \frac{1}{2}\left(x_{0, 2, 4, 6}(3) + W_8^{-3}x_{1, 3, 5, 7}(3)\right).
            \end{align*}
            由于 $3 > N' / 2 = 2$,故
            \begin{align*}
                x_{0, 2, 4, 6}(3) = \frac{1}{2}\left(x_{0, 4}(1) - W_{4}^{-3}x_{2, 6}(1)\right), \\
                x_{1, 3, 5, 7}(3) = \frac{1}{2}\left(x_{1, 5}(1) - W_{4}^{-3}x_{3, 7}(1)\right).
            \end{align*}
            因此,
            \begin{align*}
                x(3) & = \frac{1}{2}\left(\frac{1}{2}\left(x_{0, 4}(1) - W_{4}^{-3}x_{2, 6}(1)\right)
                    + W_8^{-3}\cdot \frac{1}{2}\left(x_{1, 5}(1) - W_{4}^{-3}x_{3, 7}(1)\right)\right) \\
                & = \frac{1}{4}x_{0, 4}(1) - \frac{W_{4}^{-3}}{4}x_{2, 6}(1) + \frac{W_8^{-3}}{4}x_{1, 5}(1) - \frac{W_8^{-3}W_{4}^{-3}}{4}x_{3, 7}(1).
            \end{align*}
    \end{enumerate}
\end{solution}

\begin{exercise}
    带限周期信号 $f(t)$ 的截止频率为 $700\;\mathrm{Hz}$,周期为 $0.01\;\mathrm{s}$,
    对 $f(t)$ 进行采样。
    \begin{enumerate}[label=(\arabic*)]
        \item 若采样频率为 $1200\;\mathrm{Hz}$,试用原信号频谱 $F(\omega)$ 表示出
            采样信号频谱 $\hat{F}(\omega)$ 在 $450\;\mathrm{Hz} \sim 550\;\mathrm{Hz}$ 处
            的取值。
        \item 若以 $1500\;\mathrm{Hz}, 3000\;\mathrm{Hz}$ 为采样频率分别对 $f(t)$ 进行
            采样,采样时长分别为 $0.02\;\mathrm{s}, 0.01\;\mathrm{s}$。两个采样序列
            的离散傅里叶变换 DFT 分别为 $X_1(k), X_2(k)$,其中 DFT 点数与序列长度相同。
            请从以下两个任务中选择一个完成:
            \begin{enumerate}[label=(\alph*)]
                \item 用 $X_1(k)$ 表示 $X_2(k)$。
                \item 用 $X_2(k)$ 表示 $X_1(k)$。
            \end{enumerate}
    \end{enumerate}
\end{exercise}

\begin{solution}
    \begin{enumerate}[label=(\arabic*)]
        \item 由于 $\omega_s = 1200\;\mathrm{Hz} < 2\omega_M = 1400\;\mathrm{Hz}$,故会发生混叠。
            发生混叠的部分在 $[500\;\mathrm{Hz}, 700\;\mathrm{Hz}]$ 区域,因此
            \begin{align*}
                \hat{F}(\omega) = \begin{cases}
                    \frac{1}{T_s}F(\omega), & 450\;\mathrm{Hz} \le \omega < 500\;\mathrm{Hz}, \\
                    \frac{1}{T_s}\left(F(\omega) + F(\omega - 1200)\right), & 500\;\mathrm{Hz} \le \omega \le 550\;\mathrm{Hz},
                \end{cases}
            \end{align*}
            其中 $T_s = 1 / \omega_s = 1 / 1200\;\mathrm{s}$,因此
            \begin{align*}
                \hat{F}(\omega) = \begin{cases}
                    1200F(\omega), & 450\;\mathrm{Hz} \le \omega < 500\;\mathrm{Hz}, \\
                    1200\left(F(\omega) + F(\omega - 1200)\right), & 500\;\mathrm{Hz} \le \omega \le 550\;\mathrm{Hz}.
                \end{cases}
            \end{align*}
        \item 由题知
            \begin{align*}
                L_1 = 1500\;\mathrm{Hz} \cdot 0.02\;\mathrm{s} = 30, \quad
                L_2 = 3000\;\mathrm{Hz} \cdot 0.01\;\mathrm{s} = 30.
            \end{align*}
            因此 $N = L_1 = L_2 = 30$。设 $T_1 = 0.02\;\mathrm{s} = 2T_2$。
            由于 $f(t)$ 的周期为 $0.01\;\mathrm{s}$,故
            \begin{align*}
                x_1(n) & = [f(0), f(T_1), \cdots, f(14T_1), f(0), f(T_1), \cdots, f(14T_1)], \\
                x_2(n) & = [f(0), f(T_2), \cdots, f(14T_2), f(15T_2), f(16T_2), \cdots, f(29T_2)].
            \end{align*}
            因此
            \begin{align*}
                X_2(k) = \sum_{n = 0}^{N - 1}f(nT_2)W_N^{nk}.
            \end{align*}
            对于 $X_1(k)$,有
            \begin{align*}
                X_1(k) & = \sum_{n = 0}^{N - 1}f(nT_1)W_N^{nk} \\
                & = \sum_{n = 0}^{N/2 - 1}f(nT_1)\left(W_N^{nk} + W_N^{(n+N/2)k}\right) \\
                & = [1 + (-1)^k]\sum_{n = 0}^{N/2 - 1}f(2nT_2)W_N^{nk}.
            \end{align*}
            考虑将其扩展为 $N$ 项,有
            \begin{align*}
                X_1(k) & = [1 + (-1)^k]\sum_{n = 0}^{N/2 - 1}f(2nT_2)W_N^{kn} \\
                & = [1 + (-1)^k]\sum_{n = 0}^{N - 1}\left[f(nT_2)W_N^{kn/2} \cdot \frac{1 + (-1)^n}{2}\right] \\
                & = [1 + (-1)^k]\left[\frac{1}{2}\sum_{n = 0}^{N - 1}f(nT_2)W_N^{kn/2}
                    + \frac{1}{2}\sum_{n = 0}^{N - 1}f(nT_2)W_N^{(N+k)n/2}\right] \\
                & = \frac{1 + (-1)^k}{2}\left[X_2\left(\frac{k}{2}\right) + X_2\left(\frac{k + N}{2}\right)\right].
            \end{align*}
    \end{enumerate}
\end{solution}

\begin{remark}
    更进一步地,由于采样频率 $\omega_{s2} = 3000\;\mathrm{Hz}$,
    而 $f(t)$ 的截止频率 $\omega_M = 700\;\mathrm{Hz}$,
    故 $(700\;\mathrm{Hz}, 2300\;\mathrm{Hz})$ 区间内的频谱为 $0$。
    
    由于 $\Omega_s = 1/T_{s2} = 100\;\mathrm{Hz}$,
    这意味着,当 $k \in [8, 22]$ 时,$X_2(k) = 0$。因此,$X_1(k)$ 能被化简为
    \begin{align*}
        X_1(k) = \begin{cases}
            X_2(k/2), & k = 0, 2, \cdots, 14, \\
            X_2(k/2 + 15), & k = 16, 18, \cdots, 28, \\
            0, & k = 1, 3, \cdots, 15, 17, \cdots, 29.
        \end{cases}
    \end{align*}
\end{remark}
