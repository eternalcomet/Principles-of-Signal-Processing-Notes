\subsection{习题课 3}

\begin{exercise}
    计算题。
    \begin{enumerate}[label=(\arabic*)]
        \item 已知序列 $x_1 = [1, 6, 5, 3], x_2 = [2, 7, 5, 4, 0, 1]$,
            求它们的线卷积和 $6$ 点圆卷积。
        \item 设序列 $x(n) = [1, 2, 6, 3]$,$x(n)$ 的 $4$ 点离散傅里叶
            变换 DFT 为 $X(k)$,求 $X^2(k)$ 的 $4$ 点离散傅里叶逆变换 IDFT。
    \end{enumerate}
\end{exercise}

\begin{solution}

\end{solution}

\begin{exercise}
    设序列 $x_1(n), x_2(n)$ 的长度分别为 $L_1, L_2$,$N$ 点离散傅里叶
    变换 DFT 分别为 $X_1(k), X_2(k)$。
    \begin{enumerate}[label=(\arabic*)]
        \item 记 $x_1(n)$ 和 $x_2(n)$ 的 $N$ 点圆卷积为 $y(n)$,求证:
            \begin{align*}
                \sum_{n = 0}^{N - 1}y(n) = \left(\sum_{n = 0}^{L_1 - 1}x_1(n)\right)\left(\sum_{n = 0}^{L_2 - 1}x_2(n)\right).
            \end{align*}
        \item 当 $L_1 = 6, N = 5$ 时,若 $x_1(n) = [2, 5, 1, 0, 3, -2]$,
            求 $X_1^2(k)$ 的 $5$ 点离散傅里叶逆变换 IDFT。
        \item 当 $L_2 = 4, N = 5$ 时,若 $x_2(0) = 2, x_2(2) = \mathi$,
            且 $X_2(k)$ 是实序列,求 $X_2(k)$ 的 $5$ 点离散傅里叶变换 DFT。
    \end{enumerate}
\end{exercise}

\begin{solution}

\end{solution}

\begin{exercise}
    本课程介绍了``基 $2$ 的 IFFT 算法'',即将 $N$ 点 IDFT 运算递归地
    分解为 $N/2$ 点,$N/4$ 点,……,$2$ 点 IDFT 运算。事实上,也可以将 $N$ 点 IDFT 运算
    递归地分解为 $N/3$ 点,$N/9$ 点,……,$3$ 点 IDFT 运算,实现 ``基 $3$ 的 IFFT 算法''。

    设序列 $X(k)$ 的长度为 $N$,$X(k)$ 的 $N$ 点 IDFT 为 $x(n)$,$N$ 是 $3$ 的倍数。
    将序列 $X(k)$ 分解为 $3$ 个长度为 $N/3$ 的子序列:
    \begin{align*}
        X_0(k) = X(3k), \quad X_1(k) = X(3k + 1), \quad X_2(k) = X(3k + 2), \quad 0 \le k < N/3.
    \end{align*}
    设序列 $X_0(k), X_1(k), X_2(k)$ 的 $N/3$ 点 IDFT 分别
    为 $x_0(n), x_1(n), x_2(n)$,试用 $x_0(n), x_1(n), x_2(n)$ 表示 $x(n)$。
\end{exercise}

\begin{solution}

\end{solution}

\begin{exercise}
    本课程介绍了 DFT 的快速算法 FFT。相应地,可将 $N$ 点 IDFT 运算递归地分解
    为 $N/2$ 点,$N/4$ 点,……,$2$ 点 IDFT 运算,从而实现 IDFT 的快速计算。
    \begin{enumerate}[label=(\arabic*)]
        \item 设序列 $X(k)$ 的长度为 $N$,$N$ 为偶数,$X(k)$ 的 $N$ 点 IDFT 为 $x(n)$。$G(k)$ 是 $X(k)$ 中
            下标为偶数的元素组成的子序列,$H(k)$ 是 $X(k)$ 中下标为奇数的元素组成的子序列。
            它们的长度均为 $N/2$,对应的 $N/2$ 点 IDFT 分别为 $g(n)$ 和 $h(n)$。
            试用 $g(n)$ 和 $h(n)$ 表示 $x(n)$。        
        \item 规定 $X(k)$ 的 $m$ 点子序列 $[X(k_1), X(k_2), \cdots, X(k_m)]$ 的 $m$ 点 IDFT 可
            表示为 $x_{k_1, k_2, \cdots, k_m}(n)$。当 $n = 8$ 时,试利用 (1) 中的结果,
            用若干组 $X(k)$ 的 $2$ 点子序列的 $2$ 点 IDFT 表示 $x(3)$。
    \end{enumerate}
\end{exercise}

\begin{solution}

\end{solution}

\begin{exercise}
    带限周期信号 $f(t)$ 的截止频率为 $700\;\mathrm{Hz}$,周期为 $0.01\;\mathrm{s}$,
    对 $f(t)$ 进行采样。
    \begin{enumerate}[label=(\arabic*)]
        \item 若采样频率为 $1200\;\mathrm{Hz}$,试用原信号频谱 $F(\omega)$ 表示出
            采样信号频谱 $\hat{F}(\omega)$ 在 $450\;\mathrm{Hz} \sim 550\;\mathrm{Hz}$ 处
            的取值。
        \item 若以 $1500\;\mathrm{Hz}, 3000\;\mathrm{Hz}$ 为采样频率分别对 $f(t)$ 进行
            采样,采样时长分别为 $0.02\;\mathrm{s}, 0.01\;\mathrm{s}$。两个采样序列
            的离散傅里叶变换 DFT 分别为 $X_1(k), X_2(k)$,其中 DFT 点数与序列长度相同。
            请从以下两个任务中选择一个完成:
            \begin{enumerate}[label=(\alph*)]
                \item 用 $X_1(k)$ 表示 $X_2(k)$。
                \item 用 $X_2(k)$ 表示 $X_1(k)$。
            \end{enumerate}
    \end{enumerate}
\end{exercise}

\begin{solution}

\end{solution}
