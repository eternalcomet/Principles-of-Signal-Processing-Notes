\subsection{函数的正交分解}

\subsubsection{标准正交函数集}

\begin{definition}[平方可积函数]
    令 $x(t)$ 为一实函数,若
    \begin{align*}
        \int_{-\infty}^{+\infty}x^2(t)\D{t} < +\infty,
    \end{align*}
    则称 $x(t)$ 为\bd{平方可积函数},并记作 $x(t) \in L^2(\set{R})$。
    即,$L^2(\set{R})$ 表示所有平方可积函数组成的函数空间。
\end{definition}

\begin{definition}[内积]
    设 $f_1(t)$ 和 $f_2(t)$ 为两个函数,定义它们在区间 $[t_1, t_2]$ 上的\bd{内积}为
    \begin{align*}
        \ip{f_1}{f_2} = \int_{t_1}^{t_2}f_1(t)f_2^*(t)\D{t}.
    \end{align*}
\end{definition}

\begin{remark}
    回忆第 1 章中正交函数与正交函数集的定义,可以发现,$f_1(t)$ 和 $f_2(t)$ 在 $[t_1, t_2]$ 上正交,
    是指它们在 $[t_1, t_2]$ 上\bd{互不含有对方的分量}。函数 $f_1$ 和 $f_2$ 在 $[t_1, t_2]$ 上正交
    的充要条件是它们的内积为零。即,
    \begin{align*}
        \ip{f_1}{f_2} = 0.
    \end{align*}
\end{remark}

\begin{definition}[标准函数集]
    如果在区间 $[t_1, t_2]$ 上,函数集 $\{\varphi_i(t)\}$ 满足
    \begin{align*}
        \ip{\varphi_i}{\varphi_i} = \int_{t_1}^{t_2}\varphi_i(t)\varphi_i^*\D{t}
            = \int_{t_1}^{t_2}\|\varphi_i(t)\|^2\D{t} = 1,
    \end{align*}
    则称此函数集 $\{\varphi_i(t)\}$ 为\bd{标准函数集}。
\end{definition}

\begin{definition}[标准正交函数集]
    若正交函数集 $\{\varphi_i(t)\}$ 是一个标准函数集,则称之为\bd{标准正交函数集}。
\end{definition}

\subsubsection{函数的正交分解}

\begin{definition}[函数的正交分解]
    当函数 $f(t)$ 在 $[t_1, t_2]$ 区间具有连续的一阶导数和逐段连续的二阶导数时,
    $f(t)$ 可以用完备的正交函数集 $\{\varphi_i(t)\}$ 来表示,即
    \begin{align*}
        f(t) = \sum_{i = 1}^{+\infty}c_i\varphi_i(t),
    \end{align*}
    其中 $c_i$ 为常数,则称此表示为函数 $f(t)$ 的\bd{正交分解}。

    值得注意的事,$c_i$ 可以显式地表达为
    \begin{align*}
        c_i = \frac{\ip{f}{\varphi_i}}{\ip{\varphi_i}{\varphi_i}}
            = \frac{\ip{f}{\varphi_i}}{k_i}
            = \frac{1}{k_i}\int_{t_1}^{t_2}f(t)\varphi_i^*(t)\D{t},
    \end{align*}
    而 $k_i$ 为 $\varphi_i(t)$ 在区间 $[t_1, t_2]$ 上与自己的内积,即
    \begin{align*}
        k_i = \ip{\varphi_i}{\varphi_i} = \int_{t_1}^{t_2}\varphi_i(t)\varphi_i^*(t)\D{t}
        = \int_{t_1}^{t_2}\|\varphi_i(t)\|^2\D{t}.
    \end{align*}
\end{definition}

\begin{theorem}[帕斯瓦尔定理]
    用一个正交函数集来准确地表示一个信号时,这信号的能量等于相应的正交函数各分量的能量之和。
    即,设 $f(t)$ 为一个信号,将用正交函数集 $\{\varphi_i(t)\}$ 来表示,$c_i, k_i$ 定义如上,
    则有
    \begin{align*}
        \int_{t_1}^{t_2}\|f(t)\|^2\D{t} = \sum_{i = 1}^{+\infty}\|c_i\|^2k_i.
    \end{align*}
    这被称为\bd{帕斯瓦尔(Parseval)定理}。
\end{theorem}

\begin{proof}
    由于 $\{\varphi_i(t)\}$ 是一个正交函数集,故有
    \begin{align*}
        \ip{\varphi_i}{\varphi_j} = \begin{cases}
            1, & i = j, \\
            0, & i \neq j.
        \end{cases}
    \end{align*}
    而 $f(t)$ 的正交分解为 $f(t) = \sum_{i = 1}^{+\infty}c_i\varphi_i(t)$,故有
    \begin{align*}
        \int_{t_1}^{t_2}\|f(t)\|^2\D{t} & = \int_{t_1}^{t_2}\left\|\sum_{i = 1}^{+\infty}c_i\varphi_i(t)\right\|^2\D{t} \\
        & = \int_{t_1}^{t_2}\sum_{i = 1}^{+\infty}\|c_i\varphi_i(t)\|^2\D{t}
            + \sum_{i, j \ge 1, i \neq j}\|c_ic_j^*\ip{\varphi_i}{\varphi_j}\| \\
        & = \int_{t_1}^{t_2}\sum_{i = 1}^{+\infty}\|c_i\|^2\cdot \|\varphi_i(t)\|^2\D{t} + 0 \\
        & = \sum_{i = 1}^{+\infty}\|c_i\|^2k_i.
    \end{align*}
\end{proof}

\begin{note}
    在推导的过程中,对于内积运算,别忘了\bd{取共轭}。例如上述证明中的 $c_j^*$。
\end{note}
