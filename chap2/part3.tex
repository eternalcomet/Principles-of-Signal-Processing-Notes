\subsection{信号的正交变换}

\subsubsection{信号的级数展开}

\begin{definition}[信号的级数展开]
    考虑使用一组函数 $\{\varphi_i(t)\}$,将信号 $x(t) \in L^2(\set{R})$ 展开成级数,即
    \begin{align*}
        x(t) = \sum_{i = -\infty}^{+\infty}c_i\varphi_i(t),
    \end{align*}
    这一形式称为信号 $x(t)$ 的\bd{级数展开}。

    通常,展开系数 $c_i$ 使用信号 $x(t)$ 的某种积分形式来确定。这一积分公式(即求展开系数的公式)
    称之为\bd{信号变换}。
\end{definition}

\subsubsection{函数的正交变换}

\begin{definition}[函数的正交变换]
    若信号级数展开的基函数 $\{\varphi_i(t)\}$ 为标准完备正交函数集,则积分变换
    \begin{align*}
        c_i = \int_{t_1}^{t_2}x(t)\varphi_i^*(t)\D{t}
    \end{align*}
    称为信号 $x(t)$ 的\bd{正交变换},亦称为 \bd{Karhunen-Loève 变换}。
\end{definition}

\begin{note}
    如果信号为符合狄义赫利(Dirichlet)条件的周期函数,则正交分解的系数 $c_i$ 的形式会很漂亮。
\end{note}

\subsubsection{周期函数的正交分解}

